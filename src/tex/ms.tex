% Define document class
\documentclass[reprint,superscriptaddress,nobibnotes,amsmath,amssymb,aps,hidelinks]{revtex4-2}
\usepackage{showyourwork}
\usepackage[version=4]{mhchem}
\usepackage{graphicx}
\usepackage{bm}
\usepackage{siunitx}
\sisetup{list-units=bracket,
         list-final-separator={, and },
         range-units=bracket,
         range-phrase={, },
         uncertainty-mode=separate
         }
\begin{document}

\title{My Amazing Reproducible Publication}

\author{Andrew R. McCluskey}
\email{andrew.mccluskey@bristol.ac.uk}
  \affiliation{Centre for Computational Chemsitry, School of Chemistry, University of Bristol, Cantock's Close, Bristol, BS8 1TS, UK}
  \affiliation{European Spallation Source ERIC, Ole Maaløes vej 3, 2200 København N, DK}
  \affiliation{Diamond Light Source, Harwell Campus, Didcot, OX11 0DE, UK}

% Abstract with filler text
\begin{abstract}
    Lorem ipsum dolor sit amet, consectetuer adipiscing elit.
    Ut purus elit, vestibulum ut, placerat ac, adipiscing vitae, felis.
    Curabitur dictum gravida mauris, consectetuer id, vulputate a, magna.
    Donec vehicula augue eu neque, morbi tristique senectus et netus et.
    Mauris ut leo, cras viverra metus rhoncus sem, nulla et lectus vestibulum.
    Phasellus eu tellus sit amet tortor gravida placerat.
    Integer sapien est, iaculis in, pretium quis, viverra ac, nunc.
    Praesent eget sem vel leo ultrices bibendum.
    Aenean faucibus, morbi dolor nulla, malesuada eu, pulvinar at, mollis ac.
    Curabitur auctor semper nulla donec varius orci eget risus.
    Duis nibh mi, congue eu, accumsan eleifend, sagittis quis, diam.
    Duis eget orci sit amet orci dignissim rutrum.
\end{abstract}

\maketitle

% Main body with filler text
\section{Introduction}
\label{sec:intro}

Lorem ipsum dolor sit amet, consectetuer adipiscing elit.
Ut purus elit, vestibulum ut, placerat ac, adipiscing vitae, felis.
Curabitur dictum gravida mauris, consectetuer id, vulputate a, magna.
Donec vehicula augue eu neque, morbi tristique senectus et netus et.
Mauris ut leo, cras viverra metus rhoncus sem, nulla et lectus vestibulum.
Phasellus eu tellus sit amet tortor gravida placerat.
Integer sapien est, iaculis in, pretium quis, viverra ac, nunc.
Praesent eget sem vel leo ultrices bibendum.
Aenean faucibus, morbi dolor nulla, malesuada eu, pulvinar at, mollis ac.
Curabitur auctor semper nulla donec varius orci eget risus.
Duis nibh mi, congue eu, accumsan eleifend, sagittis quis, diam.
Duis eget orci sit amet orci dignissim rutrum.

Nam dui ligula, fringilla a, euismod sodales, sollici- tudin vel, wisi.
Morbi auctor lorem non justo, nam lacus libero, pretium at, lobortis vitae.
Donec aliquet, tortor sed accumsan bibendum, erat ligula aliquet magna.
Morbi ac orci et nisl hendrerit mollis, suspendisse ut massa, cras nec ante.
Pellentesque a nulla cum sociis natoque penatibus et magnis dis parturient.
Aliquam tincidunt urna, nulla ullamcorper vestibulum turpis.
Pellentesque cursus luctus mauris \citep{Luger2021}.

%
\begin{figure}
    \centering
    \includegraphics[width=\columnwidth]{figures/my_plot.pdf}
    \script{fit-and-plot.py}
    \caption{(a) The estimated $D^{*}_{\mathrm{Li^+}}$ as a function of temperature (black points; shown with \SI{95}{\percent} credible interval error bar). The blue shading shows the corresponding posterior distribution $p[\bm{\theta}_{\alpha} | D^{*}_{\mathrm{Li^+}}(T)]$ of compatible Arrhenius parameters with the estimated $D^*_{\mathrm{Li^+}}$ data. The variegated shading indicates compatibility intervals of \numlist{1;2;3} $(\sigma\{p[\bm{\theta}_{\alpha} | D^{*}_{\mathrm{Li^+}}(T)]\})$. The marginal posterior distributions $p[E_a | D^*_{\mathrm{Li^+}}(T)]$ (b) and $p[A | D^{*}_{\mathrm{Li^+}}(T)]$ (c) obtained from the posterior distribution of Arrhenius parameters in (a).}
    \label{fig:my_plot}
\end{figure}
%

\bibliography{bib}

\end{document}
